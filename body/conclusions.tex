\cleardoublepage
\chapter*{Conclusions}
\markboth{Conclusions}{Conclusions}
\addcontentsline{toc}{chapter}{Conclusions}
\vspace{1cm}

In this work we explored and improved the applications of Koopmans spectral functionals in the context of extended systems. More specifically, we first consolidated the theoretical framework filling the gaps that hindered the use of Koopmans functionals in extended periodic systems; secondly, we devised a scheme to overcome the technical difficulties and compute the band structure of crystalline materials. Finally, we used the developed tools to carry out band structure calculations on a set of benchmark semiconductors and insulators, which confirmed the high accuracy of Koopmans spectral functionals in the prediction of the spectral properties of materials.

The conceptual developments of this thesis include the proof of the validity of Bloch's theorem in the context of Koopmans functionals and, more generally, of orbital-density-dependent functionals. While this aspect is trivially fulfilled by standard density-functional approaches, in orbital-density-dependent methods the localized character of the orbitals brings to non-periodic potentials whose compliance with Bloch's theorem is less evident. Here we showed that, with the assumption of a Wannier-like nature for the orbitals that minimize the energy functional, Bloch's theorem is still preserved, and a band structure description of the energy spectrum is possible. As an aside, the reported proof required the introduction of a novel and useful definition of the Koopmans Hamiltonian, that can be expressed as a single non-local operator, rather than via a set of local and orbital-density-dependent Hamiltonians.

In order to perform band structure calculations, we developed two schemes that exploit again the Wannier-like character of the variational orbitals: a first approach allows to unfold the band structure from supercell $\Gamma$-only calculations, reconstructing the connection between each energy eigenvalue and the correspondent $\bk$-point in the Brillouin zone; the second method maps the problem into the primitive cell, and allows to compute the band structure without the need of an unfolding procedure. Both methods were successfully used to calculate the electronic bands of semiconductors and insulators, displaying an accuracy that is comparable to that of state-of-the-art many-body perturbation theory methods. Ultimately, for band structure calculations, the current implementations of Koopmans functionals, can be considered as a ``cheap'', yet accurate, alternative to more complex electronic-structure methods.

Possible future developments of Koopmans spectral functionals can be divided in three categories: theoretical aspects, technical improvements, and applications. At the theoretical level, it would be interesting to get more insights about the actual meaning of the generalized piecewise-linearity -- or Koopmans' -- condition. Originally, Koopmans functionals were seen as a method correcting the many-body self-interaction error, with the latter interpreted as a deviation from the generalized piecewise-linearity condition. While this argument partially works in molecules, in extended systems the duality between localized variational orbitals and delocalized canonical orbitals makes it less straightforward. An alternative way of seeing the Koopmans' condition, is as a mapping that allows to define an approximated self-energy which brings to eigenvalues having the correct meaning of quasiparticle energies. While the second interpretation is probably the most correct, it would be interesting to investigate whether the Koopmans' condition brings about some self-interaction corrections or not (that goes beyond the standard piecewise-linearity). In other words, is the generalized piecewise-linearity condition a real property of the system? We remark that understanding this aspect, might be useful also to understand if it is somehow possible to extend this condition to off-diagonal occupations, and ultimately expand the applications to metallic systems. Moreover, connected to the previous question there is another aspect, that is whether the Koopmans functional represent a physical energy. As many DFT-based methods, Koopmans functionals are assumed to inherit important properties such as the variational principle. While in the limit of fully occupied and empty states the Koopmans energy tends to  the base functional (although its derivatives do not), in the general case ($f_i \neq 0,1$) it rather represents an ensemble of different excited-state energies. It would be useful then, to give a more rigorous justification for the existence of a variational principle for Koopmans functionals.

Among the several technical improvements that could raise the level of the current implementations of Koopmans functionals, we report one that is linked to the insights captured during this work. As mentioned already, the Wannier character of the variational orbitals is a sufficient condition for the compliance of the crystal Hamiltonian with the translation symmetry of the system. In standard DFT, the analogous condition is represented by the periodicity of the total density, which is then assumed \emph{a priori} in calculations on periodic systems. Similarly, in Koopmans functionals the Wannier-like nature of the variational orbitals could become a requirement, that guarantees the compliance with Bloch's theorem. We remark indeed that, although Wannier functions represent a solution that comply with the system's symmetry, there might be other sets of orbitals that are energetically equivalent, yet not symmetry-compliant. Eventually, an unconstrained minimization could lead to these orbitals rather than to Wannier functions, impeding to obtain the band structure of the system. A ``Wannier-constrained'' minimization  could provide an effective way to compute self-consistent band structures from orbital-density-dependent functionals.

Finally, backed by the recent developments and the automatization of the computational procedures, Koopmans functionals could be employed more consistently to perform calculations in extended systems. Starting from high-throughput band structure calculations (possibly coupled with some recent automatic Wannierization techniques), one could consider systems that are particularly relevant in, e.g., photovoltaic applications: perovskites, low-dimensional materials, etc. An application that was considered towards the end of this work is that of materials that contain defects. The preliminary study reported in this work showed promising results in this context and, given the computational relevance of this type of systems (electronic devices, quantum computing, etc.), and the difficulty to tackle the problem with other high-level electronic-structure methods, should be definitely considered for future applications.
