\cleardoublepage
\chapter{Introduction\label{ch:introduction}}

\section{The many-electron problem}
In materials science, all the properties of a given system whether, they are mechanical, chemical, or optical, can be explained in terms of the fundamental interactions between their elemental components: electrons and nuclei. At a good level of approximation, where only relativistic and hyperfine effects are neglected, all these interactions are embodied in the following Hamiltonian (given in atomic units):
%
\begin{equation}
    \hat{H}_{tot} = - \sum_I \frac{\nabla_I^2}{2M_I} - \sum_i \frac{\nabla_i^2}{2} + \sum_{i < j} \frac{1}{|\br_i - \br_j|} - \sum_{i,I} \frac{Z_I}{|\br_i - \bR_I|} + \sum_{I < J} \frac{Z_I Z_J}{|\bR_I - \bR_J|} ,
    \label{eq:many-body}
\end{equation}
%
where the first two terms represent the kinetic energies of the nuclei and of the electrons respectively, and the following three terms embed their individual and reciprocal electrostatic interactions.

\begin{equation}
    \hat{H} \Psi = \left(\hat{T} + \hat{W} + \hat{V} \right) \Psi = E \Psi
    \label{eq:many-electron-problem}
\end{equation}

\section{First-principles methods}

\section{Spectral properties}
- about the correlation discuss single-reference methods (orthonormality of the wave functions) vs real many-body functions which are never single Slater determinants

- problem of DFT (or KS-DFT) is that explicit expressions of energies and potentials in terms of the density are not known --> density-matrix, wave function, one-particle spin-orbitals or other stuff provide explicit expressions.

- Green's function natural generalization of density and density-matrix. Spectral properties are resolved in frequency space, then Green's function good because it has frequency. (see page 2 review Reining). Tracing interactions resulting at different spaces and times is very difficult with a local static object such as the density (reason for the difficulty in finding explicit expressions for observables in terms of the density). One also wants to avoid to express (expectation values of) observables in terms of the wave function, which have a trivial definition but that can be very complicate to handle because of the super complex character of the many-body wave function. This complexity is also a consequence of the fact that the many-body wave function ultimately carries more information than what we need. Therefore many observables, especially spectral properties, can be defined in terms of the Green's function.

\section{Koopmans spectral functionals}

\section{Objectives}

\section{Organisation of the thesis}

