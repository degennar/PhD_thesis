\cleardoublepage
\chapter[Band structures of semiconductors and insulators]{Band structures of semiconductors\break and insulators\label{ch:band-structures}}
In this chapter we discuss the details of practical calculations, and the results obtained for a manifold of reference insulating materials. Thanks to the validity of Bloch's theorem, discussed in \cref{ch:koopmans-periodic}, it is possible to obtain electronic band structures -- within the primitive cell's BZ -- either from a supercell approach, by means of an unfolding method, or from the primitive cell implementation described in \cref{sec:koopmans-pbc}. The computational details of the calculations, including the description of the unfolding method and of the Koopmans workflow, are discussed in \cref{sec:calculations-koopmans}, while in \cref{sec:results-bands} we report the obtained results.

Part of the content of this chapter, as well as the reported results, have been published in Refs.~\cite{de_gennaro_blochs_2022,colonna_koopmans_2022}.

\cleardoublepage
\section{Calculations with Koopmans functionals\label{sec:calculations-koopmans}}


\subsection{Unfolding method\label{sec:unfolding-method}}

\begin{figure}
    \centering
    \includegraphics[width=\linewidth]{WF-pcell-scell.pdf}
    \caption[]{}
    \label{fig:map-wf}
\end{figure}


\subsection{Workflow\label{sec:workflow}}

\begin{figure}
    \centering
    \subfloat[]{\includegraphics[width=\linewidth]{dscf_workflow.png}} \\
    \subfloat[]{\includegraphics[width=\linewidth]{dfpt_workflow.png}}
    \caption[]{}
    \label{fig:workflow}
\end{figure}

\subsection{Computational details\label{sec:computational-details}}

\section{Results\label{sec:results-bands}}

