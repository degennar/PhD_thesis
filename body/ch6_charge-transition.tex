\cleardoublepage
\chapter{Impurity levels of point defects\label{ch:defects}}
In this chapter, we present the preliminary study of defect levels in semiconductors with Koopmans spectral functionals. 

\clearpage
\section{Motivation}
Many properties of materials are strongly influenced by the presence of impurities and point defects. Electrical and optical properties of semiconductors can be both quenched and stimulated as a consequence of the presence of defects. In extrinsic semiconductors, the hole and electron conductivities can be finely controlled by tuning the concentration of $p$-type impurity atoms -- also called acceptors, as they trap an electron and free a hole close to the top of the valence band -- and $n$-type impurity atoms -- also called donors, as they release an electron at the bottom of the conduction band. On the other hand, the presence of defect centers can also have a reversed effect and trap charge carriers in localized states, as for gold impurities in silicon \cite{corsetti_negative-u_2014}, decreasing the conductivity of the material. More recently, the properties of impurity centers have been considered also in the context of quantum information, since they can represent optimal systems for quantum communication: the infamous nitrogen-vacancy (NV) center in diamond is one of them \cite{maze_properties_2011}, as it provides a very coherent optical transition that can be exploited to create an entangled state (a qubit of information). It is apparent, that many modern electronic and optoelectronic devices, are somehow affected by the presence of defects -- negatively, or positively. Understanding, and being able to properly simulate such effects, is then an extremely relevant topic in computational materials science.

To test the performances of Koopmans functionals for the prediction of the position of impurity levels in semiconductors, we considered the EL2 defect in gallium arsenide. The EL2 defect has been for many years a pivotal research topic due to its influence on the electrical and optical properties of GaAs, and for its appearance during the growth process of melt GaAs, despite the total absence of any doping elements. It was observed experimentally that the EL2 concentration increases with the stoichiometry ratio of the elemental species As/Ga \cite{kaminska_el2_1987}, which hinted at a connection with the As atom. Whether the EL2 defect is associated with a substitutional As-antisite -- taking the place of a Ga vacancy -- or with a complex of As-antisite with As-interstitial, was object of debate for a while; today, the interpretation of the simple As-antisite (\asga) impurity is commonly accepted. The EL2 defect level is then given by the presence of a neutral As-antisite and lies 0.75~\si{\milli\electronvolt} below the bottom of the conduction band \cite{kaminska_el2_1987,dabrowski_isolated_1989}. The positively charged state ($\rm As_{Ga}^+$) lies instead 0.5~\si{\milli\electronvolt} above the top of the valence band. Given the simple nature of this defect -- the neutral As-antisite defect state is a fully symmetric ($A_1$) singlet -- and the excellent description of the band structure of GaAs from Koopmans functionals (see \cref{ch:band-structures}), the \asga~represents a perfect test case.


 

\section{Theoretical schemes}

\section{Results}





