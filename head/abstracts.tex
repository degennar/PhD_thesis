\begingroup
\let\cleardoublepage\clearpage

% English abstract
\cleardoublepage
\chapter*{Abstract}
\markboth{Abstract}{Abstract}
\vspace{2.5cm}
%

Electronic-structure simulations have been impacting the study of materials properties thanks to the simplicity of density-functional theory, a method that gives access to the ground state of the system. Although very important, ground-state properties represent just part of the information, and often technological applications rely more on excited-state properties. In the context of density-functional theory, the latter are difficult to extract and one usually has to resort to more sophisticated approaches. In the last years, Koopmans spectral functionals have emerged as an effective method which combines the feasibility of density-functional theory with the accuracy of more complex methods, such as many-body perturbation theory. While retaining its simplicity, Koopmans functionals extend the domain of density-functional theory providing direct access to charged excitations, and ultimately to the photoemission spectra of materials.

This approach has been extensively employed in finite systems, displaying an accuracy which is comparable to that of state-of-the-art many-body perturbation theory methods. In extended systems, calculations were bound to the supercell ($\Gamma$-only) method, preventing the access to the full band structure of the system. In this work we overcome this limitation, proving that a band structure description of the energy spectrum is possible, and providing a scheme to carry out calculations in crystalline materials.

The first result of this work consists in proving the compliance of Koopmans functionals with the translation symmetry of the system. The validity of Bloch's theorem, thus the possibility of describing the spectrum via a band structure picture, depends on this condition. Because of the orbital-density-dependent nature of the functional, the invariance of the total energy with respect to unitary transformations of the one-electron orbitals is broken. The energy is then minimized by a particular set of orbitals, called ``variational'', which are strongly localized in space. In extended periodic systems, the localized, thus non-periodic, character of the variational orbitals is inherited by the effective orbital-density-dependent Hamiltonians, which apparently break the translation symmetry of the system. Here we show that, by requiring the variational orbitals to be Wannier functions, the translation symmetry is preserved and Bloch's theorem holds.

In the second part, we devise a scheme to unfold the band structure from supercell ($\Gamma$-only) calculations, and reconstruct the $\bk$-dependence of the quasiparticle energies. This method is then used to compute the band structures of a set of benchmark semiconductors and insulators. Finally, we describe a novel formulation of Koopmans functionals for extended periodic systems, which exploits from the beginning the translation properties of Wannier functions to realize a primitive cell-based implementation of Koopmans functionals. Results obtained from this second approach are also discussed.

In the last part, we present the preliminary study of impurity states arising in crystalline materials in the presence of point defects.

%
\paragraph{Keywords:}
Koopmans spectral functionals,
orbital-density-dependent functionals
spectral properties,
photoemission spectra,
Bloch's theorem,
band structure,
band gap,
point-defects,
impurity states.
%
\addcontentsline{toc}{chapter}{Abstract} % adds an entry to the table of contents

% Italian abstract
\begin{otherlanguage}{italian}
\cleardoublepage
\chapter*{Sommario}
\markboth{Sommario}{Sommario}
\vspace{2.4cm}
%

Le simulazioni di struttura elettronica hanno avuto un impatto sullo studio delle propriet\`{a} dei materiali grazie alla semplicit\`{a} della teoria del funzionale densit\`{a}, un metodo che d\`{a} accesso allo stato fondamentale del sistema. Sebbene molto importanti, le propriet\`{a} di stato fondamentale rappresentano solo una parte dell'informazione e spesso le applicazioni tecnologiche si basano maggiormente sulle propriet\`{a} di stato eccitato. Nel contesto della teoria del funzionale densit\`{a}, questi ultimi sono difficili da estrarre e di solito si deve ricorrere ad approcci più sofisticati. Negli ultimi anni, i funzionali spettrali Koopmans sono emersi come un metodo efficace che combina la fattibilit\`{a} della teoria del funzionale densit\`{a} con l'accuratezza di metodi più complessi, come la teoria delle perturbazioni a molti corpi. Pur mantenendo la sua semplicit\`{a}, i funzionali di Koopmans estendono il dominio della teoria del funzionale densit\`{a} fornendo un accesso diretto alle eccitazioni cariche e, in definitiva, agli spettri di fotoemissione dei materiali.

Questo approccio \`{e} stato ampiamente utilizzato in sistemi finiti, mostrando un'accuratezza paragonabile a quella dei metodi più avanzati nella teoria delle perturbazioni a molti corpi. In sistemi estesi, i calcoli sono stati finora vincolati al metodo della supercella (solo $\Gamma$), impedendo l'accesso all'intera struttura a bande del sistema. In questo lavoro superiamo questa limitazione, dimostrando che \`{e} possibile una descrizione dello spettro energetico  per mezzo della struttura a bande, e fornendo uno schema per eseguire calcoli in materiali cristallini.

Il primo obiettivo di questo lavoro consiste nel dimostrare il rispetto delle simmetrie di traslazione del sistema da parte dei funzionali Koopmans. La validit\`{a} del teorema di Bloch, quindi la possibilit\`{a} di descrivere lo spettro tramite la struttura a bande, dipende da questa condizione. A causa della dipendenza del funzionale dalle densit\`{a} orbitali, l'invarianza dell'energia totale rispetto alle trasformazioni unitarie degli orbitali elettronici viene a mancare. L'energia viene quindi minimizzata da un particolare insieme di orbitali, detti variazionali, che sono spazialmente molto localizzati. In sistemi periodici estesi, il carattere localizzato, quindi non periodico, degli orbitali variazionali viene ereditato dalle Hamiltoniane effettive, che sembrano rompere la simmetria traslazionale del sistema. In questo lavoro di tesi mostriamo che, richiedendo che gli orbitali variazionali siano funzioni di Wannier, la simmetria di traslazione viene preservata e il teorema di Bloch \`{e} valido.

Nella seconda parte, elaboriamo uno schema per ottenere la struttura a bande della cella primitiva dai calcoli in supercella (solo $\Gamma$) e ricostruire la dipendenza da $\bk$ delle energie di quasiparticella. Questo metodo viene quindi utilizzato per calcolare le strutture a bande di un insieme di semiconduttori e isolanti di riferimento. Infine, descriviamo una nuova formulazione dei funzionali Koopmans per sistemi periodici estesi, che sfrutta fin dall'inizio le propriet\`{a} di traslazione delle funzioni di Wannier per realizzare un'implementazione in cella primitiva dei funzionali Koopmans. Vengono discussi anche i risultati ottenuti da questo secondo approccio.

Nell'ultima parte, presentiamo lo studio preliminare degli stati di impurezza che si manifestano nei materiali cristallini in presenza di difetti puntuali.

%
\paragraph{Parole chiave:}
funzionali spettrali Koopmans,
funzionali dipendenti dalle densit\`{a} orbitali,
propriet\`{a} spettrali,
spettri di fotoemissione,
teorema di Bloch,
struttura a bande,
banda proibita,
difetti puntuali,
stati di impurezza.
%
\addcontentsline{toc}{chapter}{Sommario} % adds an entry to the table of contents
\end{otherlanguage}

\endgroup			
\vfill
