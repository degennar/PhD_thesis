\begingroup
\let\cleardoublepage\clearpage

% English abstract
\cleardoublepage
\chapter*{Abstract}
\markboth{Abstract}{Abstract}
\vspace{2cm}
%

In the last years, Koopmans spectral functionals have emerged as a competitive method for the description of the direct and inverse photoemission processes in materials. This approach has been extensively employed in finite systems, displaying an accuracy which is comparable to that of state-of-the-art many-body perturbation theory methods. In extended systems, calculations were bound to the supercell ($\Gamma$-only) method, preventing the access to the full band structure of the system. In this work we overcome this limitation, proving that a band structure description of the energy spectrum is possible, and providing a scheme to carry out calculations in crystalline materials.

The first result of this work consists in proving the compliance of Koopmans functionals with the translation symmetry of the system. The validity of Bloch's theorem, thus the possibility of describing the spectrum via a band structure picture, depends on this condition. Because of the orbital-density-dependent nature of the functional, the invariance of the total energy with respect to unitary transformations of the one-electron orbitals, is broken. The energy is then minimized by a particular set of orbitals, called variational, which are strongly localized in space. In extended periodic systems, the localized, thus non-periodic, character of the variational orbitals is inherited by the effective orbital-density-dependent Hamiltonians, which apparently break the translation symmetry of the system. Here we show that, by requiring the variational orbitals to be Wannier functions, the translation symmetry is preserved and Bloch's theorem holds.

In the second part, we devise a scheme to unfold the band structure from supercell ($\Gamma$-only) calculations, and reconstruct the $\bk$-dependence of the quasiparticle energies. This method is then used to compute the band structures of a set of benchmark semiconductors and insulators. Finally, we describe a novel formulation of Koopmans functionals for extended periodic systems, which exploits from the beginning the translation properties of Wannier functions to realize a primitive cell-based implementation of Koopmans functionals. Results obtained from this second approach are also discussed.

In the last part, we present the preliminary study of impurity states arising in crystalline materials in the presence of point defects.

%
\paragraph{Keywords:}
Koopmans spectral functionals,
orbital-density-dependent functionals
spectral properties,
photoemission spectra,
Bloch's theorem,
band structure,
band gap,
point-defects,
impurity states.
%
\addcontentsline{toc}{chapter}{Abstract} % adds an entry to the table of contents

% Italian abstract
\begin{otherlanguage}{italian}
\cleardoublepage
\chapter*{Sommario}
\markboth{Sommario}{Sommario}
\vspace{2cm}
%
Testo qui
%
\paragraph{Parole chiave:}
funzionali spettrali Koopmans,
funzionali Koopmans-compilant
%
\addcontentsline{toc}{chapter}{Sommario} % adds an entry to the table of contents
\end{otherlanguage}

\endgroup			
\vfill
