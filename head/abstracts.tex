\begingroup
\let\cleardoublepage\clearpage

% English abstract
\cleardoublepage
\chapter*{Abstract}
\markboth{Abstract}{Abstract}
\vspace{2cm}
%

Koopmans spectral functionals have emerged in the last years as a competitive method for the description of the direct and inverse photoemission processes in materials. Koopmans functionals have been extensively employed in finite systems, exhibiting an accuracy comparable to that of state-of-the-art many-body perturbation theory methods. In extended systems, calculations were so far limited to $\Gamma$-only supercell calculations, preventing the access to the full band structure of the material . In this work we identify the limitations affecting extended systems 

Koopmans spectral functionals represent a class of orbital-density-dependent functionals that aim to describe the electron addition and removal processes. The orbital-density-dependent character resulting from the imposition of a generalized piecewise-linearity condition -- also called Koopmans' condition -- brings to a more complex framework with respect to standard density-functional theory. At the same time, it provides a bridge between variational density-functional approaches and many-body perturbation theory methods, which ultimately justifies the interpretation of Koopmans functionals as a spectral theory. The applications of Koopmans functionals to finite systems displayed a remarkable accuracy in the prediction of photoemission spectra, competing with that of state-of-the-art many-body perturbation theory methods.

In this work, we focus on the applications of Koopmans spectral functionals to extended systems. 

The first band gap calculations confirmed the high level of accuracy observed in molecules. However, the need for a localized set of variational orbitals (those that minimize the energy functional) forced to resort to the supercell method, that impeded obtain the band structure of the 


%
\paragraph{Keywords:}
Koopmans spectral functionals,
Koopmans-compliant functionals
%
\addcontentsline{toc}{chapter}{Abstract} % adds an entry to the table of contents

% Italian abstract
\begin{otherlanguage}{italian}
\cleardoublepage
\chapter*{Sommario}
\markboth{Sommario}{Sommario}
\vspace{2cm}
%
Testo qui
%
\paragraph{Parole chiave:}
funzionali spettrali Koopmans,
funzionali Koopmans-compilant
%
\addcontentsline{toc}{chapter}{Sommario} % adds an entry to the table of contents
\end{otherlanguage}

\endgroup			
\vfill
