\chapter{Koopmans for metallic systems\label{app:koopmans-metals}}
In this section we point out the issues arising in metallic systems, or anytime that the occupation number matrix is non-diagonal. We start from the spectral representation of the density operator
%
\begin{equation}
    \hat{\rho} = \sum_i f_i \ket{\psi_i} \bra{\psi_i} \ ,
    \label{eq:density-matrix-diag}
\end{equation}
%
where $\{ \psi_i \}$ is the set of KS eigenvectors, and the occupation numbers follow the Fermi-Dirac distribution: $f_i=1$ for the occupied states, and $f_i=0$ for the empty states. By representing $\hat{\rho}$ on the basis of $\{ \psi_i \}$ we obtain the following form for the occupation number matrix:

\begin{equation}
    \mathcal{F} = 
    \left(\vphantom{\begin{array}{c}1\\1\\1\\1\\1\\1\\1\end{array}}
    \smash{\overbrace{
        \begin{array}{cccc}
            1      & 0      & \cdots & 0      \\
            0      & 1      &        & 0      \\
            \vdots &        & \ddots & \vdots \\
            0      & 0      & \cdots & 1      \\
            0      & 0      & \cdots & 0      \\
            \vdots &        &        & \vdots \\
            0      & 0      & \cdots & 0      \\
        \end{array}
        }^{N}} \ 
        \smash{\overbrace{
        \begin{array}{ccc}
            0      & \cdots & 0      \\
            0      &        & 0      \\
            \vdots &        & \vdots \\
            0      & \cdots & 0      \\
            0      & \cdots & 0      \\
            \vdots & \ddots & \vdots \\
            0      & \cdots & 0      \\
        \end{array}
        }^{M-N}}
    \right)
    = \ \mymathbb{1}_{\rm occ} \ \oplus \ \mymathbb{0}_{\rm emp} ,
    \label{eq:occ-num-matrix}
\end{equation}

where $M$ is the dimension of the full Hilbert space, and $N$ is that of the subspace $\hat{\rho}$. In a compact way, $\mathcal{F}$ can be expressed as the direct sum of two matrices defined in the $M$-dimensional Hilbert space: $\mymathbb{1}_{\rm occ}$ that is 0 everywhere and acts as the identity matrix over the $N$-dimensional occupied subspace, and $\mymathbb{0}_{\rm emp}$ that is 0 everywhere and it acts as the null matrix over the $(M-N)$-dimensional empty subspace.

Let's consider now a change of representation from the set of KS states $\{ \psi_i \}$ to a set of orbitals $\{ \phi_i \}$, e.g. the variational orbitals. The two basis are connected by a unitary transformation $U: \ket{\phi_i} = \sum_j U_{ji} \ket{\psi_j}$. In terms of the new vectors, the density operator reads as
%
\begin{equation}
    \hat{\rho} = \sum_{jk} \tilde{f}_{jk} \ket{\phi_j} \bra{\phi_k} \ ,
    \label{eq:density-matrix-off-diag}
\end{equation}
%
where $\tilde{f}_{jk}$ are the matrix elements of $\tilde{\mathcal{F}} = U^{\dagger} F U$, that in general is a non-diagonal matrix. The presence of off-diagonal elements in the density matrix represents a problem: the Koopmans' condition [see \cref{eq:koopmans-condition}] applies only to the orbital occupations or, in other words, to the diagonal elements of $\mathcal{F}$. As a consequence, the formalism of Koopmans functionals does not contain any term regarding off-diagonal occupations ($f_{ij}$), or mixed orbital densities ($phi_i^*(\br) \phi_j(\br)$), terms which then are not treated within the current theory.

Nevertheless, some particular choices for the transformation $U$ can avoid this problem and preserve the diagonal form of $\hat{\rho}$. Thanks to its block diagonal form, the $\mathcal{F}$-matrix in \eqref{eq:occ-num-matrix} turns out to be invariant over transformations that do not mix occupied and empty states. These cases are represented by a unitary matrix $U$ having the same block diagonal form of $\mathcal{F}$:

\begin{equation}
    U = 
    \begin{pmatrix}
        U_{\rm occ} & 0 \\
        0 & U_{\rm emp}
    \end{pmatrix}
     = \mathbb{U}_{\rm occ} \oplus \mathbb{U}_{\rm emp} \ ;
     \label{eq:U-matrix-separate}
\end{equation} \\
%
since both the identity matrix and the null matrix are trivially invariant over unitary transformations, the $\mathcal{F}$-matrix does not change, preserving also the diagonal form of \cref{eq:occ-num-matrix}. Eventually, this is the way we proceed when we compute the variational orbitals of Koopmans functionals, whether we determine them self-consistently, or we use a non-self-consistent guess. 

Solving this problem would open to the application of Koopmans functionals to metallic systems, impossible to tackle otherwise. Moreover, also for insulating systems this would facilitate the applications in periodic systems, since it would allow computing MLWFs without the constraint of \cref{eq:U-matrix-separate}, which forces the separate Wannierization of the occupied and empty manifolds (usually more complex than the Wannierization of the full manifold).
